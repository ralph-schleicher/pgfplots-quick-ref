%% qr-body.tex  -*- TeX-master: "qr.tex" -*-

% Copyright (C) 2018 Ralph Schleicher

% Permission is granted to copy, distribute, and/or modify this document
% under the terms of the GNU Free Documentation License, Version 1.3 or
% any later version published by the Free Software Foundation; with no
% Invariant Sections, no Front-Cover Texts, and no Back-Cover Texts.
%
% You should have received a copy of the GNU Free Documentation License
% along with this document.  If not, see <https://www.gnu.org/licenses/>.

%% Code:

\chapter{General}


\section{Document Structure}

\begin{multicols}{2}
\begin{verbatim}
\documentclass{standalone}
\usepackage{pgfplots}
\pgfplotsset{compat=1.16}

\begin{document}
\begin{tikzpicture}
\begin{axis}[title={$y = e^x$}]
\addplot+[no markers] {exp(x)};
\end{axis}
\end{tikzpicture}
\end{document}
\end{verbatim}
\columnbreak
\begin{tikzpicture}
\begin{axis}[
  width=\the\linewidth,
  title={$y = e^x$},
]
\addplot+[
  no markers,
] {exp(x)};
\end{axis}
\end{tikzpicture}
\end{multicols}


\section{\PGFPlots{} Options}

\texindex{pgfplotsset}
\begin{meta}
\\pgfplotsset\{\metavar{key/value list}\}
\metavar{key/value list}\=\group{\metavar{key} = \metavar{value},}\*
\end{meta}
%\index{option prefix!pgfplots@\protect\code{/pgfplots/}}
%\index{pgfplots!option prefix@\protect\code{/pgfplots/}}
%\index{option prefix!tikz@\protect\code{/tikz/}}
%\index{tikz!option prefix@\protect\code{/tikz/}}
Options are supplied as a \metavar{key/value list}.
The \code{/pgfplots/} and \code{/tikz/} prefixes in \metavar{key} can
be omitted in the scope of \PGFPlots{} commands.  Please note that a
trailing comma in \metavar{key/value list} does no harm.


\section{Key Handlers}

\index{handler|see {key handler}}
\index{style option|see {key handler}}
\index{code option|see {key handler}}

\label{key:/.style}
\begin{meta}
\\pgfplotsset\{\metavar{key}/.style = \{\metavar{key/value list}\}\}
\end{meta}
\index{key handler!style@\protect\code{.style}}
\index{style key handler@\protect\code{.style} key handler}
Define or replace style \metavar{key}.

\label{key:/.append style}
\begin{meta}
\\pgfplotsset\{\metavar{key}/.append style = \{\metavar{key/value list}\}\}
\end{meta}
\index{key handler!append style@\protect\code{.append style}}
\index{append style key handler@\protect\code{.append style} key handler}
Append to style \metavar{key}.

\label{key:/.code}
\begin{meta}
\\pgfplotsset\{\metavar{key}/.code = \{\metavar{\TeX{} code}\}\}
\end{meta}
\index{key handler!code@\protect\code{.code}}
\index{code key handler@\protect\code{.code} key handler}
Define or replace \metavar{key} that -- when run -- takes one argument;
\metavar{\TeX{} code} can refer to the supplied argument as \code{\#1}.
Invoke as ‘\code{\\pgfplotsset\{\metavar{key} = \{\metavar{argument}\}\}}’.

\label{key:/.code 2 args}
\begin{meta}
\\pgfplotsset\{\metavar{key}/.code 2 args = \{\metavar{\TeX{} code}\}\}
\end{meta}
\index{key handler!code@\protect\code{.code 2 args}}
\index{code 2 args key handler@\protect\code{.code 2 args} key handler}
Like \hyperref[key:/.code]{\code{\metavar{key}/.code}}
but with two arguments; \metavar{\TeX{} code} can refer to the
supplied arguments as \code{\#1} and \code{\#2}.  Invoke as
‘\code{\\pgfplotsset\{\metavar{key} = \{\metavar{first argument}\}\{\metavar{second argument}\}\}}’.

\label{key:/.cd}
\begin{meta}
\\pgfplotsset\{\metavar{key}/.cd\}
\end{meta}
\index{key handler!cd@\protect\code{.cd}}
\index{cd key handler@\protect\code{.cd} key handler}
Make \metavar{key} the default prefix.


\section{Mathematical Expressions}

See the \TikZ/\PGF{} manual for a detailed description.

Use parenthesis, \code{(} and \code{)}, for grouping.  Arguments and
values of trigonometric functions are in degree angle.

\begingroup
\def\item[#1]{{\sl #1\/}:}
\item[Arithmetic Operators]
\code{+}, \code{-}~(also unary minus), \code{*}, \code{/},
\code{\textasciicircum}~(exponentiation),
\code{!}~(factorial, postfix operator),
\code{r}~(radian, postfix operator, see \code{deg}).

\item[Relational Operators]
\code{==}, \code{!=}, \code{<}, \code{<=}, \code{>}, \code{>=}.

\item[Logical Operators]
\code{!}~(not, prefix operator),
\code{||}~(or), \code{\&\&}~(and).

\item[Conditionals]
\code{\metavar{condition}?\metavar{true}:\metavar{false}}.

\item[Constants]
\code{pi}, \code{e}, \code{false}, \code{true}.

\item[Unary Functions]
\code{abs}, \code{sign}, \code{int}, \code{frac}~(fractional part),
\code{round}, \code{floor}, \code{ceil},
\code{factorial} (see \code{!}),
\code{iseven}, \code{isodd}, \code{isprime},
\code{sqrt},
\code{exp}, \code{ln}, \code{log10}, \code{log2},
\code{sin}, \code{cos}, \code{tan}, \code{cot}, \code{sec}, \code{cosec},
\code{asin}, \code{acos}, \code{atan},
\code{deg}~(degree from radian), \code{rad}~(radian from degree),
\code{sinh}, \code{cosh}, \code{tanh}.

\item[Binary Functions]
\code{div}~(integer division), \code{mod}, \code{Mod}~(unsigned result),
\code{gcd},
\code{pow}~(see \code{\textasciicircum}),
\code{atan2},
\code{veclen}~(vector length in~$\mathbb{R}^2$).

\item[$n$-ary Functions]
\code{min}, \code{max}.

\item[Pseudo-Random Number Functions (Uniform Distribution)]
\code{rnd}~($[0,1]\cap\mathbb{R}$),
\code{rand}~($[-1,1]\cap\mathbb{R}$),
\code{random($n$)}~($[1,n]\cap\mathbb{N}$),
\code{random($m$,$n$)}~($[m,n]\cap\mathbb{Z}$).
\endgroup


\chapter{Axis Environments}

\envindex{axis}
\envindex{semilogxaxis}
\envindex{semilogyaxis}
\envindex{loglogaxis}
\begin{meta}
\\begin\{axis\}[\metavar{axis options}]\?
\metavar{axis options}\=\metavar{key/value list}
\end{meta}
\env{axis} can also be \env{semilogxaxis}, \env{semilogyaxis}, or
\env{loglogaxis}.

\optindex{every linear axis}
\optindex{every semilogx axis}
\optindex{every semilogy axis}
\optindex{every loglog axis}
\begin{meta}
\defsty{every \metavar{type}\? axis}
\metavar{type}\=\group{linear\|semilogx\|semilogy\|loglog}
\end{meta}
Define default axis options.

\optindex{xmode}
\optindex{ymode}
\optindex{zmode}
\symindex{normal}
\symindex{linear}
\symindex{log}
\begin{meta}
\defopt{xmode\|ymode\|zmode} = \default{normal}\|linear\|log
\end{meta}
\index{axis scaling}
\index{linear!axis scaling}
\index{logarithmic!axis scaling}
Customize axis scaling; \code{linear} is a synonym for \code{normal}.

\optindex{log basis}
\begin{meta}
\defopt{log basis \group{x\|y\|z}} = \default{\metavar{empty}}\|\metavar{positive number}
\end{meta}
\index{axis scaling!basis for logarithm}
The basis for logarithmic axis scaling.  Empty means to apply the
natural logarithm (base~$e$) to any input coordinate -- if the axis
scaling is logarithmic -- and use the decadic/common logarithm
(base~10) for displaying tick labels.  Any non-empty value causes
both, coordinates and tick labels, to use the logarithm with base
\metavar{number}.


\chapter{Plots}

\texindex{addplot}
\begin{meta}
\\addplot[\metavar{plot options}]\? \metavar{input data} \metavar{trailing TikZ path commands};
\end{meta}
\code{\\addplot} (without options) and
\code{\\addplot+[\metavar{plot options}]} utilize default options from the
cycle list.
\code{\\addplot[\metavar{plot options}]} only use the manually provided
options.

\optindex{every axis plot}
\begin{meta}
\defsty{every axis plot \group{no \var{n}}\?}
\end{meta}
Define \metavar{plot options} for all plots or for the $n^{\rm th}$ plot
of every axis.  Plot numbers are zero-based.


\section{Input Data}

\optindex{empty line}
\symindex{auto}
\symindex{none}
\symindex{scanline}
\symindex{jump}
\begin{meta}
\defopt{empty line} = \default{auto}\|none\|scanline\|jump
\end{meta}
How to handle empty lines in \metavar{coordinates list}, \code{none}
means to do nothing, \code{jump} means to insert a discontinuity.


\subsection{Coordinates List}

\symindex{coordinates}
\symindex{+-}
\begin{meta}
\metavar{input data}\=coordinates \{\metavar{coordinates list}\}
\metavar{coordinates list}\=\metavar{coordinates}\*
\metavar{coordinates}\=(\var{x}, \var{y}, \var{z}) \group{+- (\var{u}, \var{v}, \var{w})}\? \group{[\metavar{meta data}]}\?
\end{meta}
\index{input data!coordinates list}
\index{coordinates list!input data}
\index{list of coordinates!input data}
\index{sequence of coordinates!input data}
Read input data from a sequence of coordinates.
\var{x}, \var{y}, and \var{z} are the point coordinates.
\var{u}, \var{v}, and \var{w} are the error coordinates (reliability
bounds) for error bar plots.  Coordinate \var{z} and \var{w} are only
mandatory for \acronym{3D} plots.
Empty lines in the \metavar{coordinates list} indicate
discontinuities; use \code{\\\\} when gathering coordinates in a
\TeX{} macro.

\suboptindex{plot coordinates/}{math parser}
\begin{meta}
\defopt{plot coordinates/math parser} = \default{true}\|false
\end{meta}
Whether or not to enable mathematical expressions in every coordinate
inside of a \metavar{coordinates list}.


\subsection{Table Data}

\symindex{table}
\begin{meta}
\metavar{input data}\=table [\metavar{table options}]\? \{\metavar{table data}\}
\metavar{table data}\={\metavar{file name}\|\metavar{inline table}}
\end{meta}
\index{input data!table data}
\index{table data!input data}
Read input data from table columns.

\suboptindex{table/}{x}
\suboptindex{table/}{y}
\suboptindex{table/}{z}
\suboptindex{table/}{x error}
\suboptindex{table/}{x error plus}
\suboptindex{table/}{x error minus}
\suboptindex{table/}{y error}
\suboptindex{table/}{y error plus}
\suboptindex{table/}{y error minus}
\suboptindex{table/}{z error}
\suboptindex{table/}{z error plus}
\suboptindex{table/}{z error minus}
\suboptindex{table/}{meta}
\begin{meta}
\defopt{\implicit{table/}\metavar{coordinate}} = \metavar{column name}
\defopt{\implicit{table/}\metavar{coordinate} index} = \metavar{column index}
\defopt{\implicit{table/}\metavar{coordinate} expr} = \metavar{expression}
\metavar{coordinate}\=x\|y\|z\|\group{x\|y\|z} error \group{plus\|minus}\?\|meta
\end{meta}
Column names are case sensitive and have to exist.  Use
\code{\{\metavar{column name}\}} to quote non-trivial column names.
The first column has index zero.

\texindex{thisrow}
\texindex{thisrowno}
\texindex{coordindex}
\texindex{lineno}
Within \metavar{expression} \code{\\thisrow\{\metavar{column name}\}}
and \code{\\thisrowno\metavar{column index}} yields the cell value of
the specified column.  Likewise, \code{\\coordindex} yields the index
of the current set of coordinates and \code{\\lineno} yields the total
line number.  Both numbers start counting at zero.

\suboptindex{table/}{header}
\begin{meta}
\defopt{\implicit{table/}header} = \default{true}\|false
\end{meta}
Whether or not to check \metavar{table data} for column names.
If enabled, the first non-comment line is checked for column names.
That means if any element is not a number, all entries are treated as
column names.

\suboptindex{table/}{skip first n}
\begin{meta}
\defopt{\implicit{table/}skip first n} = \default{0}\|\metavar{non-negative integer}
\end{meta}
Don't process the first~$n$ lines in \metavar{table data}.

\suboptindex{table/}{ignore chars}
\suboptindex{table/}{white space chars}
\suboptindex{table/}{comment chars}
\begin{meta}
\defopt{\implicit{table/}ignore chars} = \default{\{\}}\|\metavar{comma-separated list}
\defopt{\implicit{table/}white space chars} = \default{\{\}}\|\metavar{comma-separated list}
\defopt{\implicit{table/}comment chars} = \default{\{\}}\|\metavar{comma-separated list}
\end{meta}
Extra characters to be ignored, treated like a whitespace character
(beside space and tab), or treated like a comment start character
(beside \code{\#} and \code{\%}).

\suboptindex{table/}{row sep}
\begin{meta}
\defopt{\implicit{table/}row sep} = \default{\metavar{newline}}\|\\\\
\end{meta}
Use \code{\\\\} as the row seperator if you experience problems with
\metavar{newline}, for example with inline table data or when
gathering table data in a \TeX{} macro.

\suboptindex{table/}{col sep}
\begin{meta}
\defopt{\implicit{table/}col sep} = \default{space}\|tab\|comma\|semicolon\|colon\wrap\|braces\|&\|ampersand
\end{meta}
A \code{space} column separator means one or more space or tab
characters.  With \code{braces}, every table cell looks like
\code{\{\metavar{contents}\}} and whitespace characters between
adjacent table cells is ignored.  A \code{\&} column separator
implies ‘\code{table/trim cells = true}’.

\suboptindex{table/}{read completely}
\begin{meta}
\defopt{\implicit{table/}read completely} = \default{auto}\|true\|false
\end{meta}
Whether or not to read the whole table into memory.
Use with care!

\suboptindex{table/}{search path}
\subsuboptindex{table/}{search path/}{implicit .}
\begin{meta}
\defopt{\implicit{table/}search path} = \default{\{\}}\|\metavar{comma-separated list}
\defopt{\implicit{table/}search path/implicit .} = \default{true}\|false
\end{meta}
Search path for input files, \code{.} means to use the standard
\TeX{} procedure.

\texindex{pgfplotstableread}
\begin{meta}
\\pgfplotstableread\{\metavar{file name}\}\\foo
\\addplot table [\metavar{table options}] \{\\foo\};
\end{meta}
Read table data once so that you can use it multiple times;
\code{\\foo} is a user-defined command sequence.


\subsection{Mathematical Expressions}

\symindex{expression}
\begin{meta}
\metavar{input data}\={expression}\? \{\metavar{expression}\}
\metavar{input data}\=(\metavar{x-expression}, \metavar{y-expression}, \metavar{z-expression})
\end{meta}
Create input data by sampling a mathematical expression over an
argument domain.  The second form can be used to create parametric
plots.  Say \code{\{\metavar{x-expression}\}} if
\metavar{x-expression} contains parenthesis or commas.  The
\metavar{z-expression} is only mandatory for \acronym{3D} plots.

\optindex{domain}
\optindex{domain y}
\begin{meta}
\defopt{domain} = \default{-5:5}\|\metavar{\(x\sub1\)}:\metavar{\(x\sub2\)}
\defopt{domain y} = \default{\metavar{empty}}\|\metavar{\(y\sub1\)}:\metavar{\(y\sub2\)}
\end{meta}
Define the argument domain for the x-axis to the closed interval
$[x_1, x_2]$.  Likewise for the y-axis for \acronym{3D} plots.  If
\code{domain y} is empty, use the value of \code{domain}.

\optindex{samples}
\optindex{samples y}
\begin{meta}
\defopt{samples} = \default{25}\|\metavar{non-negative integer}
\defopt{samples y} = \default{\metavar{empty}}\|\metavar{non-negative integer}
\end{meta}
The number of samples to be generated.  Samples are equally spaced
over the corresponding argument domain.  If ‘\code{samples y}’ is empty,
use the value of \code{samples}.

\optindex{samples at}
\begin{meta}
\defopt{samples at} = \default{\{\}}\|\metavar{comma-separated list of numbers}
\end{meta}
Explicit argument values for sampling \metavar{expression}.  This
option always overrides the \code{domain} and \code{samples} options.
\metavar{comma-separated list of numbers} can contain \code{...}
expressions, for example ‘\code{\{-2, -1.8, ..., 2\}}’.

\optindex{variable}
\optindex{variable y}
\begin{meta}
\defopt{variable} = \default{x}\|\metavar{variable name}
\defopt{variable y} = \default{y}\|\metavar{variable name}
\end{meta}
The variable name containing the argument value when evaluating
\metavar{expression}.


\section{Line Plots}

\begin{footnotesize}
\begin{tabulary}{\linewidth}{@{} c @{\hskip\columnsep} c @{\hskip\columnsep} c @{\hskip\columnsep} c @{}}
\demoplot{tikz/addplot-line-sharp.tikz} &
\demoplot{tikz/addplot-line-smooth.tikz} &
\demoplot{tikz/addplot-line-const.tikz} &
\demoplot{tikz/addplot-line-jump.tikz} \\
\code{sharp plot} &
\code{smooth} &
\code{const plot} &
\code{jump mark left} \\
\end{tabulary}
\end{footnotesize}

\optindex{sharp plot}
\begin{meta}
\defopt\tikzkey{sharp plot}
\end{meta}
Connect points by straight lines.  This is the default.

\optindex{smooth}
\optindex{tension}
\begin{meta}
\defopt\tikzkey{smooth}
\defopt\tikzkey{tension} = \default{0.55}\|\metavar{number}
\end{meta}
Connect points by a smooth curve.  For best results, points should be
equidistant and the bending angles should be less than about~$30^\circ$.
The \code{tension} option controls the sharpness of the corners; $0$
yields sharp corners and $1$ yields a circle if the path is a square.

\optindex{const plot}
\optindex{const plot mark left}
\optindex{const plot mark mid}
\optindex{const plot mark right}
\begin{meta}
\defopt\tikzkey{const plot}
\defopt\tikzkey{const plot mark \group{\default{left}\|mid\|right}}
\end{meta}
Connect points with horizontal and vertical line segments.
‘\code{const plot}’ is an alias for ‘\code{const plot mark left}’.
Markers are placed on the left corner, in the middle, or on the right
corner of the horizontal line segments.
Use ‘\code{const plot, no markers}’ to omit the markers.

\optindex{jump mark left}
\optindex{jump mark mid}
\optindex{jump mark right}
\begin{meta}
\defopt\tikzkey{jump mark \group{left\|mid\|right}}
\end{meta}
Like ‘\code{const plot}’ but omit the vertical line segments.


\section{Bar Plots}

\begin{footnotesize}
\begin{tabulary}{\linewidth}{@{} c @{\hskip\columnsep} c @{\hskip\columnsep} c @{\hskip\columnsep} c @{}}
\demoplot{tikz/addplot-bar-xbar.tikz} &
\demoplot{tikz/addplot-bar-ybar.tikz} &
\demoplot{tikz/addplot-bar-xbar-interval.tikz} &
\demoplot{tikz/addplot-bar-ybar-interval.tikz} \\
\code{xbar} &
\code{ybar} &
\code{xbar interval} &
\code{ybar interval} \\
\end{tabulary}
\end{footnotesize}

\optindex{xbar}
\optindex{ybar}
\begin{meta}
\defopt\tikzkey{xbar}
\defopt\tikzkey{ybar}
\end{meta}
Render coordinates as horizontal or vertical bars respectively.

\optindex{bar width}
\begin{meta}
\defopt{\pgfkey{bar width}} = \default{10pt}\|\metavar{dimension}\|\metavar{number}
\end{meta}
\texindex{pgfplotbarwidth}
Width of a single bar.  \metavar{dimension} is a \TeX{} dimension and
\metavar{number} is in axis units.  Value can be a mathematical
expression.  The fully computed value is then available in
\code{\\pgfplotbarwidth}.

\optindex{bar shift}
\begin{meta}
\defopt{\pgfkey{bar shift}} = \default{0pt}\|\metavar{dimension}\|\metavar{number}
\end{meta}
\texindex{pgfplotbarshift}
Off-center distance for the bars.  \metavar{dimension} is a \TeX{}
dimension and \metavar{number} is in axis units.  Value can be a
mathematical expression.  The fully computed value is then available
in \code{\\pgfplotbarshift}.

\styindex{xbar}
\styindex{ybar}
\begin{meta}
\defsty{xbar}
\defopt{xbar}\group{ = \default{2pt}\|\metavar{dimension}\|\metavar{number}}\?
\defsty{ybar}
\defopt{ybar}\group{ = \default{2pt}\|\metavar{dimension}\|\metavar{number}}\?
\end{meta}
Predefined axis style for bar plots; implies \code{\tikzkey{xbar}}
or \code{\tikzkey{ybar}} respectively, \sty{bar shift auto}, and
\sty{bar cycle list}.  The default handler takes one optional
argument which is passed on to \sty{bar shift auto}.

\styindex{bar shift auto}
\begin{meta}
\defsty{bar shift auto}
\defopt{bar shift auto} = \default{2pt}\|\metavar{dimension}\|\metavar{number}
\end{meta}
Predefined axis style setting \pgfkey{bar shift} to the correct value
based on the current plot number and the total number of plots.
Argument is the distance between adjacent bars of a group.

When $n$ bar plots are added to an axis, the total width for a group of
bars is $n\times\metavar{bar width}+(n-1)\times\metavar{bar shift auto}$.

\styindex{bar cycle list}
\begin{meta}
\defsty{bar cycle list}
\end{meta}
Predefined axis style installing a cycle list for bar plots.

\optindex{bar direction}
\begin{meta}
\defopt{bar direction} = \default{auto}\|x\|y
\end{meta}
Explicitly set the bar plot direction.  Not needed if you say, for
example ‘\code{ybar, bar width = 1}’, because the direction is clear
from the context.

\optindex{xbar interval}
\optindex{ybar interval}
\begin{meta}
\defopt\tikzkey{xbar interval}
\defopt\tikzkey{ybar interval}
\end{meta}
Like \tikzkey{xbar} or \tikzkey{ybar} respectively, but draw the bar
width as an interval from this point to the next point.  You need one
extra point to define the interval for the last bar.

\styindex{xbar interval}
\styindex{ybar interval}
\begin{meta}
\defsty{xbar interval}
\defopt{xbar interval}\group{ = \default{1}\|\metavar{relative width}}\?
\defsty{ybar interval}
\defopt{ybar interval}\group{ = \default{1}\|\metavar{relative width}}\?
\end{meta}
Predefined axis style for interval bar plots; implies
\code{\tikzkey{xbar interval}} or \code{\tikzkey{ybar interval}}
respectively and \sty{bar cycle list}.  The default handler takes one
optional argument to scale the intervals.

\styindex{xticklabel interval boundaries}
\styindex{yticklabel interval boundaries}
\styindex{zticklabel interval boundaries}
\begin{meta}
\defsty{xticklabel interval boundaries}
\defsty{yticklabel interval boundaries}
\defsty{zticklabel interval boundaries}
\end{meta}
Axis style to display the interval bounds in the tick labels.


\section{Comb Plots}

\begin{footnotesize}
\begin{tabulary}{\linewidth}{@{} c @{\hskip\columnsep} c @{\hskip\columnsep} c @{\hskip\columnsep} c @{}}
\demoplot{tikz/addplot-comb-xcomb.tikz} &
\demoplot{tikz/addplot-comb-ycomb.tikz} & & \\
\code{xcomb} &
\code{ycomb} & & \\
\end{tabulary}
\end{footnotesize}

\optindex{xcomb}
\optindex{ycomb}
\begin{meta}
\defopt\tikzkey{xcomb}
\defopt\tikzkey{ycomb}
\end{meta}
Render coordinates as horizontal or vertical lines respectively.


\section{Quiver Plots}

\optindex{quiver}
\begin{meta}
\defopt{quiver} = \{\metavar{quiver options}\}
\end{meta}
Render coordinates as small arrows.  The origin of the arrow is at the
final point coordinates $(x,y,z)$ and the direction and length of the
arrow is defined by the direction coordinates $(u,v,w)$.

The \code{quiver/} prefix can be omitted
within \metavar{quiver options}.

\suboptindex{quiver/}{u}
\suboptindex{quiver/}{v}
\suboptindex{quiver/}{w}
\begin{meta}
\defopt{\implicit{quiver/}\group{u\|v\|w}} = \default{0}\|\metavar{expression}
\end{meta}
The direction coordinates of the arrows.  Within \metavar{expression},
\code{x}, \code{y}, and \code{z} are bound to the final point
coordinates.

For parametric plots use ‘\code{variable = t}’ and ‘\code{quiver/u = f(t)}’
and ‘\code{quiver/v = g(t)}’ to access the parameter.

\begin{footnotesize}
\begin{verbatim}
\addplot[
  variable = t,
  quiver = {u = {-sin(t)}, v = {cos(t)}},
]
({cos(t)}, {sin(t)});
\end{verbatim}
\end{footnotesize}

\suboptindex{quiver/}{u value}
\suboptindex{quiver/}{v value}
\suboptindex{quiver/}{w value}
\begin{meta}
\defopt{\implicit{quiver/}\group{u\|v\|w} value} = \default{0}\|\metavar{number}
\end{meta}
Like \code{quiver/u}, \code{quiver/v}, and \code{quiver/w}
respectively but without parsing mathematical expressions.  However,
\code{\\thisrow\{\metavar{column name}\}} and similar code works.

\suboptindex{quiver/}{colored}
\begin{meta}
\defopt{\implicit{quiver/}colored}
\defopt{\implicit{quiver/}colored} = \default{mapped color}\|\metavar{color}
\end{meta}
Set a different color for each arrow.  \code{quiver/colored} is an
alias for ‘\code{quiver/colored = mapped color}’.  Please note that
‘\code{\metavar{color}, quiver = {\metadots}}’ is more efficient if
\metavar{color} is constant.

\suboptindex{quiver/}{scale arrows}
\begin{meta}
\defopt{\implicit{quiver/}scale arrows} = \default{1}\|\metavar{number}
\end{meta}
Scale all arrows by a constant factor.

\suboptindex{quiver/}{update limits}
\begin{meta}
\defopt{\implicit{quiver/}update limits} = \default{true}\|false
\end{meta}
Whether or not the coordinates of the arrow heads shall be considered
when determining the axis limits.

\substyindex{quiver/}{every arrow}
\begin{meta}
\defsty{quiver/every arrow}
\end{meta}
Style to customize arrows individually at visualization time.

\subcodeindex{quiver/}{before arrow}
\subcodeindex{quiver/}{after arrow}
\begin{meta}
\defcode{quiver/before arrow}
\defcode{quiver/after arrow}
\end{meta}
Run \metavar{\TeX{} code} before and after drawing a single arrow.
Empty by default.

\substyindex{quiver/}{quiver legend}
\begin{meta}
\defsty{\implicit{quiver/}quiver legend}
\end{meta}
Style that redefines \code{legend image code} in order to produce a
suitable legend for quiver plots.

%\section{Stacked Plots}
%\section{Area Plots}
%\section{Closing Plots}
%\section{Scatter Plots}


\chapter{Lines and Markers}


\section{Line Width}

\styindex{ultra thin}
\styindex{very thin}
\styindex{thin}
\styindex{semithick}
\styindex{thick}
\styindex{very thick}
\styindex{ultra thick}
\begin{meta}
\qrlinewidth{ultra@thin}
\qrlinewidth{very@thin}
\qrlinewidth{thin}
\qrlinewidth{semithick}
\qrlinewidth{thick}
\qrlinewidth{very@thick}
\qrlinewidth{ultra@thick}
\end{meta}
\index{line width}
Predefined line widths.

\optindex{line width}
\begin{meta}
\defopt{\implicit{/tikz/}line width} = \default{0.4pt}\|\metavar{dimension}
\end{meta}
\index{line width}
Set the line width.


\section{Line Cap}

\optindex{line cap}
\begin{meta}
\defopt{\implicit{/tikz/}line cap} = \default{butt}\|rect\|round
\end{meta}
Set the line cap style.

\medskip
\begin{threecolumns}
\qrlinecap{butt} &
\qrlinecap{rect} &
\qrlinecap{round} \\
\end{threecolumns}


\section{Line Join}

\optindex{line join}
\begin{meta}
\defopt{\implicit{/tikz/}line join} = \default{miter}\|bevel\|round
\end{meta}
Set the line join style.

\medskip
\begin{threecolumns}
\qrlinejoin{miter} &
\qrlinejoin{bevel} &
\qrlinejoin{round} \\
\end{threecolumns}

\begin{meta}
\defopt{\implicit{/tikz/}miter limit} = \default{10}\|\metavar{number}
\end{meta}
When the ratio of the miter length to the line width is greater than
\metavar{number}, the miter join is replaced by a bevel.  A miter
limit $\ell=1/\sin(\alpha/2)$ for $\alpha\in(0^\circ,\,180^\circ]$
will create a bevel join for angles less than
$\alpha=2\cdot\arcsin(1/\ell)$.


\section{Dash Pattern}

\styindex{solid}
\styindex{dashed}
\styindex{densely dashed}
\styindex{loosely dashed}
\styindex{dotted}
\styindex{densely dotted}
\styindex{loosely dotted}
\styindex{dashdotted}
\styindex{densely dashdotted}
\styindex{loosely dashdotted}
\styindex{dashdotdotted}
\styindex{densely dashdotdotted}
\styindex{loosely dashdotdotted}
\begin{meta}
\qrlinestyle{solid}
\qrlinestyle{dashed}
\qrlinestyle{dotted}
\qrlinestyle{dashdotted}
\qrlinestyle{dashdotdotted}
\qrlinestyle{densely@dashed}
\qrlinestyle{densely@dotted}
\qrlinestyle{densely@dashdotted}
\qrlinestyle{densely@dashdotdotted}
\qrlinestyle{loosely@dashed}
\qrlinestyle{loosely@dotted}
\qrlinestyle{loosely@dashdotted}
\qrlinestyle{loosely@dashdotdotted}
\end{meta}
\index{line style}
Predefined line styles.

\optindex{dash pattern}
\begin{meta}
\defopt{\implicit{/tikz/}dash pattern} = \group{\group{on\|off} \metavar{dimension}}\+
\end{meta}
\index{line style}
\index{dash pattern}
Set the dash pattern (line style) for drawing lines, e.g.,
‘\code{dash pattern = on 3.5mm off 0.7mm}’.

\optindex{dash phase}
\begin{meta}
\defopt{\implicit{/tikz/}dash phase} = \default{0pt}\|\metavar{dimension}
\end{meta}
\index{dash phase}
Start the dash pattern at offset \metavar{dimension}.


\section{Markers}

\noindent
Standard markers:

\medskip
\begin{threecolumns}
\qrlinemarker{none}; & & \\
\qrlinemarker{*}bullet; &
\qrlinemarker{x}cross; &
\qrlinemarker{+}plus; \\
\end{threecolumns}

\noindent
With \code{\\usetikzlibrary\{plotmarks\}}:

\medskip
\begin{threecolumns}
\qrlinemarker{o}; &
\qrlinemarker{|}vert; &
\qrlinemarker{-}minus; \\
\end{threecolumns}
\begin{threecolumns}
\qrlinemarker{asterisk}; &
\qrlinemarker{star}; &
\qrlinemarker{10-pointed@star}; \\
\end{threecolumns}
\begin{threecolumns}
\qrlinemarker{Mercedes@star}; &
\qrlinemarker{Mercedes@star@flipped}; & \\
\end{threecolumns}
\begin{threecolumns}
\qrlinemarker{oplus}; &
\qrlinemarker{oplus*}; & \\
\end{threecolumns}
\begin{threecolumns}
\qrlinemarker{otimes}; &
\qrlinemarker{otimes*}; & \\
\end{threecolumns}
\begin{threecolumns}
\qrlinemarker{square}; &
\qrlinemarker{square*}; &
\qrlinemarker{halfsquare*}; \\
\end{threecolumns}
\begin{threecolumns}
\qrlinemarker{triangle}; &
\qrlinemarker{triangle*}; & \\
\end{threecolumns}
\begin{threecolumns}
\qrlinemarker{diamond}; &
\qrlinemarker{diamond*}; &
\qrlinemarker{halfdiamond*}; \\
\end{threecolumns}
\begin{threecolumns}
\qrlinemarker{pentagon}; &
\qrlinemarker{pentagon*}; & \\
\end{threecolumns}
\begin{threecolumns}
\qrlinemarker{halfcircle}; &
\qrlinemarker{halfcircle*}; & \\
\end{threecolumns}
\begin{threecolumns}
\qrlinemarker{heart}; &
\qrlinemarker{text}; & \\
\end{threecolumns}
\medskip

\noindent
All markers plotted with
‘\code{mark options = \{draw = blue, fill = yellow\}}’ and
‘\code{mark color = pink}’.  You can rotate makers with, e.g.,
‘\code{mark options = \{rotate = 90\}}’.

\optindex{mark}
\begin{meta}
\defopt{\implicit{/tikz/}mark} = \default{*}\|\metavar{marker}
\end{meta}
Use \metavar{marker}.

\optindex{mark size}
\begin{meta}
\defopt{\implicit{/tikz/}mark size} = \default{2pt}\|\metavar{dimension}
\end{meta}
Marker size, \metavar{dimension} is either the radius or about half
the width or height.

\optindex{mark repeat}
\begin{meta}
\defopt{\implicit{/tikz/}mark repeat} = \default{1}\|\metavar{integer}
\end{meta}
Draw a marker at every \metavar{integer}${}^{\rm th}$ sample.

\optindex{mark phase}
\begin{meta}
\defopt{\implicit{/tikz/}mark phase} = \default{1}\|\metavar{integer}
\end{meta}
Draw the first marker at the \metavar{integer}${}^{\rm th}$ sample;
\metavar{integer} is one based.

\optindex{mark indices}
\begin{meta}
\defopt{\implicit{/tikz/}mark indices} = \default{\{\}}\|\{\metavar{comma-separated list of integers}\}
\end{meta}
Explicit sample indices for drawing markers.
\metavar{comma-separated list of integers} can contain \code{...}
expressions, for example ‘\code{mark indices = \{1, 2, ..., 7\}}’.

\styindex{every mark}
\begin{meta}
\defsty{\implicit{/tikz/}every mark}
\end{meta}
This style is applied before drawing a marker.

\optindex{mark options}
\begin{meta}
\defopt{\implicit{/tikz/}mark options} = \{\metavar{options}\}
\end{meta}
Redefine ‘\code{every mark}’ so that it sets \metavar{options}.

\optindex{no markers}
\begin{meta}
\defsty{\implicit{/pgfplots/}no markers}
\end{meta}
Disable markers; even for cycle lists that contain markers.

\optindex{mark color}
\begin{meta}
\defopt{\implicit{/pgf/}mark color} = \default{white}\|\metavar{color}
\end{meta}
Additional fill color for \code{halfcircle}, \code{halfcircle*},
\code{halfdiamond*}, and \code{halfsquare*} markers.

\optindex{text mark}
\begin{meta}
\defopt{\implicit{/pgf/}text mark} = \default{p}\|\metavar{text}
\end{meta}
Define the text for ‘\code{mark = text}’.

\optindex{text mark as node}
\begin{meta}
\defopt{\implicit{/pgf/}text mark as node} = \default{false}\|true
\end{meta}
Whether or not to draw text markers as nodes.

\optindex{text mark style}
\begin{meta}
\defopt{\implicit{/pgf/}text mark style} = \{\metavar{options}\}
\end{meta}
Customize the appearance of text markers.
When ‘\code{text mark as node}’ is true,
‘\code{text mark style}’ are \code{\\node} options.  Otherwise,
‘\code{text mark style}’ are \code{\\pgftext} options.


\section{Colors}

Color support is provided by the \code{xcolor} package.
Standard color names:

\begin{multicols}{4}
\qrcolor{black} \\
\qrcolor{darkgray} \\
\qrcolor{gray} \\
\qrcolor{lightgray} \\
\qrcolor{white} \\
\qrcolor{red} \\
\qrcolor{cyan} \\
\qrcolor{brown} \\
\qrcolor{orange} \\
\qrcolor{teal} \\
\qrcolor{green} \\
\qrcolor{magenta} \\
\qrcolor{lime} \\
\qrcolor{pink} \\
\qrcolor{violet} \\
\qrcolor{blue} \\
\qrcolor{yellow} \\
\qrcolor{olive} \\
\qrcolor{purple} \\
\hskip 8.4pt\enskip\code{none}
\end{multicols}

\optindex{color}
\begin{meta}
\defopt{\implicit{/tikz/}color} = \metavar{color}
\end{meta}
Set the color for drawing and filling.  You can omit the option key if
\metavar{color} is a color name.

\optindex{draw}
\optindex{fill}
\begin{meta}
\defopt{\implicit{/tikz/}draw} = \metavar{color}
\defopt{\implicit{/tikz/}fill} = \metavar{color}
\end{meta}
Set the color for drawing or filling respectively.  You can use
\code{none} as \metavar{color} to disable drawing or filling.

\begin{meta}
\\definecolor\{\metavar{name}\}\{\metavar{model}\}\{\metavar{spec}\}
\metavar{model}\=rgb\|cmy\|cmyk\|hsb\|Hsb\|tHsb\|gray\|RGB\|HSB\|Gray\|HTML\wrap\|wave
\metavar{\code{rgb} spec}\=\(x\),\,\(x\),\,\(x\)
\metavar{\code{cmy} spec}\=\(x\),\,\(x\),\,\(x\)
\metavar{\code{cmyk} spec}\=\(x\),\,\(x\),\,\(x\),\,\(x\)
\metavar{\code{hsb} spec}\=\(x\),\,\(x\),\,\(x\)
\metavar{\code{Hsb} spec}\=\(H\),\,\(x\),\,\(x\)
\metavar{\code{tHsb} spec}\=\(H\),\,\(x\),\,\(x\)
\metavar{\code{gray} spec}\=\(x\)
\metavar{\code{RGB} spec}\=\(L\),\,\(L\),\,\(L\)
\metavar{\code{HSB} spec}\=\(M\),\,\(M\),\,\(M\)
\metavar{\code{Gray} spec}\=\(N\)
\metavar{\code{HTML} spec}\=\([\code{000000}\sub{\rm16},\,\code{FFFFFF}\sub{\rm16}]\)
\metavar{\code{wave} spec}\=\([363,\,814]\)
\end{meta}
$x=[0,\,1]$,
$H=[0,\,360]$,
$L=[0,\,255]\cap\mathbb{Z}$,
$M=[0,\,240]\cap\mathbb{Z}$, and
$N=[0,\,15]\cap\mathbb{Z}$.
All colors are defined in the sRGB color space.
\acronym{hsb} is a synonym for \acronym{hsl}.

%% qr-body.tex ends here
