%% qr-macros.tex  -*- TeX-master: "qr.tex" -*-

% Copyright (C) 2018 Ralph Schleicher

% Permission is granted to copy, distribute, and/or modify this document
% under the terms of the GNU Free Documentation License, Version 1.3 or
% any later version published by the Free Software Foundation; with no
% Invariant Sections, no Front-Cover Texts, and no Back-Cover Texts.
%
% You should have received a copy of the GNU Free Documentation License
% along with this document.  If not, see <https://www.gnu.org/licenses/>.

%% Code:

\makeatletter

% Special names.
\def\TikZ{\mbox{Ti\emph{k\/}Z}}
\def\PGF{{\sc pgf}}
\def\PGFPlots{{\sc pgfPlots}}
\def\PGFPlotsTable{{\sc pgfPlotsTable}}
\def\carriagereturn{\Pisymbol{psy}{191}}

% Non-cumulative font changes.
\DeclareOldFontCommand{\rm}{\normalfont\rmfamily}{\mathrm}
\DeclareOldFontCommand{\sf}{\normalfont\sffamily}{\mathsf}
\DeclareOldFontCommand{\tt}{\normalfont\ttfamily}{\mathtt}
\DeclareOldFontCommand{\bf}{\normalfont\bfseries}{\mathbf}
\DeclareOldFontCommand{\it}{\normalfont\itshape}{\mathit}
\DeclareOldFontCommand{\sl}{\normalfont\slshape}{\@nomath\sl}
\DeclareOldFontCommand{\sc}{\normalfont\scshape}{\@nomath\sc}

% Displayed text.
\newenvironment{smalldisplay}{%
\medskip
\begingroup
\footnotesize}{%
\endgroup
\medskip}

% Like ‘alltt’ but keep the normal fonts.
\newenvironment{format}{%
\begin{alltt}
\reset@font}{%
\end{alltt}}

% Simple BNF.
\def\metavar#1{\mbox{$\langle${\sl#1\/}$\rangle$}} % non-terminal
\def\metadots{\mbox{\rm\ldots}}
\def\metadef{${}\mathrel\rightarrow{}$}
\def\metaor{${}\mathbin|{}$}
\def\group#1{\mbox{$($#1$)$}}
\def\kleene#1{${}^{#1}$}
\def\overlay#1{\leavevmode
  \hbox to\z@{\hbox to\linewidth{#1}\hss}}
\def\defkey#1{\overlay{\hfil
    \usekomafont{disposition}\footnotesize#1}%
  \hbox to 1em{$\mathord\triangleright$\hfil}}
\def\defopt{\defkey{option}}
\def\defsty{\defkey{\hyperref[key:/.style]{style}}}
\def\defcode{\defkey{\hyperref[key:/.code]{code}}}
\def\implicit#1{{\color{gray}#1}}
\def\tikzkey#1{\implicit{/tikz/}#1}
\def\pgfkey#1{\implicit{/pgf/}#1}
\def\pgfplotskey#1{\implicit{/pgfplots/}#1}
\let\default\underline
\def\wrap{\,\carriagereturn\par\hskip4em\relax}

\newenvironment{meta}{%
\begin{format}
\let\=\metadef
\let\|\metaor
\def\*{\kleene*}
\def\+{\kleene+}
\def\?{\kleene?}
\let\\\textbackslash
\let\sub_
\let\sup^
\tt}{%
\end{format}
\unskip\nobreak
\vskip0.5ex\nobreak\hskip1em\relax}

% Text styles.
\def\code{\mbox\bgroup\let\\\textbackslash\tt\code@i}
\def\code@i#1{#1\egroup}
\def\var#1{$#1$}
\def\subtype#1#2{$\hbox{\tt#1}_{\scriptscriptstyle\rm#2}$}
\def\env#1{\subtype{#1}{env}}
\def\sty#1{\subtype{#1}{sty}}
\let\acronym\textsc

% Index entries; put symbols into the main index.
\def\opt@index{\sindex[opt]}
\let\sym@index\index
% Options and key handlers.
\def\optindex#1{\opt@index{#1@\protect\code{#1}}}
\def\suboptindex#1#2{\opt@index{#1@\protect\code{#1}!#2@\protect\code{#2}}}
\def\subsuboptindex#1#2#3{\opt@index{#1@\protect\code{#1}!#2@\protect\code{#2}!#3@\protect\code{#3}}}
\def\styindex#1{\opt@index{#1@\protect\sty{#1}}}
\def\substyindex#1#2{\opt@index{#1@\protect\code{#1}!#2@\protect\sty{#2}}}
\def\subsubstyindex#1#2#3{\opt@index{#1@\protect\code{#1}!#2@\protect\code{#2}!#3@\protect\sty{#3}}}
\def\codeindex#1{\opt@index{#1@\protect\subtype{#1}{code}}}
\def\subcodeindex#1#2{\opt@index{#1@\protect\code{#1}!#2@\protect\subtype{#2}{code}}}
\def\subsubcodeindex#1#2#3{\opt@index{#1@\protect\code{#1}!#2@\protect\code{#2}!#3@\protect\subtype{#3}{code}}}
% TeX control sequence.
\def\texindex#1{\sym@index{#1@\protect\code{\protect\\#1}}}
% LaTeX environment.
\def\envindex#1{\sym@index{#1@\protect\env{#1}}}
% Symbol name.
\def\symindex#1{\sym@index{#1@\protect\code{#1}}}

% Demo plots.
\def\democolumns{@{} c @{\hskip\columnsep} c @{\hskip\columnsep} c @{\hskip\columnsep} c @{}}
\def\demoplot#1{\resizebox{\fourcolumnswidth}{!}{\input#1}}
\def\demoplotx[#1]#2{\includegraphics[width=\fourcolumnswidth,#1]{#2}}

\newenvironment{twocolumns}{\footnotesize
  \tabulary{\linewidth}{@{} c @{\hskip\columnsep} c @{}}}{%
  \endtabulary}
\newenvironment{threecolumns}{\footnotesize
  \tabulary{\linewidth}{@{} c @{\hskip\columnsep} c @{\hskip\columnsep} c @{}}}{%
  \endtabulary}
\newenvironment{fourcolumns}{\footnotesize
  \tabulary{\linewidth}{@{} c @{\hskip\columnsep} c @{\hskip\columnsep} c @{\hskip\columnsep} c @{}}}{%
  \endtabulary}

\makeatother

% These colors are perceptually uniform, i.e., the primary colors
% red, green, and blue have similar lightness in the CIE L*a*b*
% color space.  Likewise for the secondary colors cyan, magenta
% and yellow.
\definecolor{unired}{HTML}{D82F00}
\definecolor{uniorange}{HTML}{DC7500}
\definecolor{uniyellow}{HTML}{D8AB00}
\definecolor{unilawn}{HTML}{7D9700}
\definecolor{unigreen}{HTML}{007C00}
\definecolor{unisea}{HTML}{00AC9B}
\definecolor{unicyan}{HTML}{27D0FF}
\definecolor{unisky}{HTML}{009EFF}
\definecolor{uniblue}{HTML}{2754FF}
\definecolor{univiolet}{HTML}{B565FF}
\definecolor{unimagenta}{HTML}{FF83FF}
\definecolor{unirose}{HTML}{FF3687}
% Gray levels with the same lightness as the primary, secondary,
% and tertiary colors.
\definecolor{unigray1}{HTML}{6C6C6C}
\definecolor{unigray2}{HTML}{B6B6B6}
\definecolor{unigray3}{HTML}{919191}

\def\qrlinestyle#1{%
  \StrSubstitute{#1}{@}{-}[\qrdashed]%
  \StrSubstitute{#1}{@}{ }[\qrspaced]%
  \overlay{\hfil\includegraphics{tikz/line-style-\qrdashed.pdf}\hskip 0.5in\relax}%
  \defsty{\implicit{/tikz/}\qrspaced}}
\def\qrlinewidth#1{%
  \StrSubstitute{#1}{@}{-}[\qrdashed]%
  \StrSubstitute{#1}{@}{ }[\qrspaced]%
  \overlay{\hfil\includegraphics{tikz/line-width-\qrdashed.pdf}\hskip 0.5in\relax}%
  \defsty{\implicit{/tikz/}\qrspaced}}
\def\qrlinecap#1{%
  \StrSubstitute{#1}{@}{-}[\qrdashed]%
  \StrSubstitute{#1}{@}{ }[\qrspaced]%
  \begin{minipage}{\threecolumnswidth}
    \centering
    \includegraphics{tikz/line-cap-\qrdashed.pdf}
    \vskip -0.25ex
    \code{\qrspaced}%
  \end{minipage}}
\def\qrlinejoin#1{%
  \StrSubstitute{#1}{@}{-}[\qrdashed]%
  \StrSubstitute{#1}{@}{ }[\qrspaced]%
  \begin{minipage}{\threecolumnswidth}
    \centering
    \includegraphics{tikz/line-join-\qrdashed.pdf}
    \vskip -0.25ex
    \code{\qrspaced}%
  \end{minipage}}
\def\qrlinemarker#1#2;{%
  \edef\qrdashed{#2}
  \ifx\qrdashed\empty
    \StrSubstitute{#1}{*}{x}[\qrdashed]%
    \StrSubstitute{\qrdashed}{@}{-}[\qrdashed]%
  \fi
  \StrSubstitute{#1}{@}{ }[\qrspaced]%
  \begin{minipage}{\threecolumnswidth}
    \centering
    \includegraphics{tikz/line-marker-\qrdashed.pdf}
    \vskip -1ex
    \code{\qrspaced}%
    \smallskip
  \end{minipage}
  \ignorespaces}
\def\qrcolor#1{%
  \StrSubstitute{#1}{@}{-}[\qrdashed]%
  \StrSubstitute{#1}{@}{ }[\qrspaced]%
  \leavevmode\hbox{\includegraphics{tikz/color-\qrdashed.pdf}\enskip\code\qrspaced}}

%% qr-macros.tex ends here
