%% qr.tex  -*- TeX-master: "qr.tex" -*-

% Copyright (C) 2018 Ralph Schleicher

% Permission is granted to copy, distribute, and/or modify this document
% under the terms of the GNU Free Documentation License, Version 1.3 or
% any later version published by the Free Software Foundation; with no
% Invariant Sections, no Front-Cover Texts, and no Back-Cover Texts.
%
% You should have received a copy of the GNU Free Documentation License
% along with this document.  If not, see <https://www.gnu.org/licenses/>.

%% Code:

% These colors are perceptually uniform, i.e., the primary colors
% red, green, and blue have similar lightness in the CIE L*a*b*
% color space.  Likewise for the secondary colors cyan, magenta
% and yellow.
\definecolor{unired}{RGB}{216, 47, 0}
\definecolor{uniorange}{RGB}{220, 118, 0}
\definecolor{uniyellow}{RGB}{216, 171, 0}
\definecolor{unilawn}{RGB}{125, 151, 0}
\definecolor{unigreen}{RGB}{0, 124, 0}
\definecolor{unisea}{RGB}{0, 172, 155}
\definecolor{unicyan}{RGB}{39, 208, 255}
\definecolor{unisky}{RGB}{0, 157, 255}
\definecolor{uniblue}{RGB}{39, 84, 255}
\definecolor{univiolet}{RGB}{181, 101, 255}
\definecolor{unimagenta}{RGB}{255, 131, 255}
\definecolor{unirose}{RGB}{255, 54, 135}
% Gray levels with the same lightness as the primary, secondary,
% and tertiary colors.
\definecolor{unigray1}{RGB}{108, 108, 108}
\definecolor{unigray2}{RGB}{182, 182, 182}
\definecolor{unigray3}{RGB}{145, 145, 145}

\def\qrcolor#1{%
\leavevmode\hbox{%
\begin{tikzpicture}[
  x = 4pt,
  y = 4pt,
]
\draw[
  fill = #1,
]
(0, 0) rectangle (2, 1);
\end{tikzpicture}
\enskip\code{#1}}}

\begin{multicols}{4}
\qrcolor{unired} \\
\qrcolor{uniorange} \\
\qrcolor{uniyellow} \\
\qrcolor{unilawn} \\
\qrcolor{unigreen} \\
\qrcolor{unisea} \\
\qrcolor{unicyan} \\
\qrcolor{unisky} \\
\qrcolor{uniblue} \\
\qrcolor{univiolet} \\
\qrcolor{unimagenta} \\
\qrcolor{unirose} \\
\qrcolor{unigray1} \\
\qrcolor{unigray3} \\
\qrcolor{unigray2} \\
\qrcolor{unigray3}
\end{multicols}

%% qr-galley.tex ends here
